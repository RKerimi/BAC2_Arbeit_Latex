\chapter{Einleitung}\label{chap:Einleitung}
\thispagestyle{standard}
\pagestyle{standard}
\renewcommand{\footrulewidth}{0.4pt}
\lfoot{\small Refik Kerimi}

Der steigende Energieverbrauch, der sich laut Statistik Austria (Stand 2016) seit den 70er Jahren fast verdoppelt hat, erfordert ein Umdenken in der Planung und Bereitstellung verschiedenster Energiequellen \cite{StatistikAustria}. Die Vernetzung und Digitalisierung unserer Welt lassen den Stromverbrauch immer rasanter steigen.
Neue und nachhaltige Stromerzeuger, wie z.B.: Photovoltaikanlagen, werden in unserer Gesellschaft immer wichtiger. 
Ein Problem stellt allerdings die Zusammenführung Intelligenter und Simpler Energiegeräte dar.
Die bestehende Infrastruktur reicht nicht aus, um Energiesysteme großflächig und gezielt steuern zu können.
Diese steigenden Anforderungen setzen eine Weiterentwicklung der Kommunikations- und Informationsinfrastruktur voraus. 

\section{Motivation} \label{sub:Motivation}
Die zunehmende Anzahl an dezentralen Stromerzeugungssystemen (engl. \acl{DER}, \acs{DER}) führt zu einem Umdenken bei der Planung und beim Betrieb von elektrischen Anlagen. Intelligente Stromnetze (engl. Smart Grids) sollen den Wandel von einem klassischen Verteilernetz, mit wenigen zentralen Großerzeugern, zu einem Netz mit vielen dezentralen Erzeugern ermöglichen und dadurch zu einer effizienteren Nutzung der Stromnetzinfrastruktur beitragen \cite{DERs, SmartGrids}.\\
Zurzeit besteht noch kein einheitliches System zur Integration von \acs {DER} Steuerungen in dem Smart Grid Bereich, da die Skalierbarkeit und Offenheit in den benötigten Automatisierungssystemen fehlt.
Ein weiteres Problem stellen die vielen unterschiedlichen Kommunikationsprotokolle dar.
Um Automatisierungs- und Steuerungsaufgaben in Smart Grids einfach und wiederverwendbar einzugliedern wird von \acl{AIT} (\acs{AIT}), FH Salzburg und Fronius International GmbH OpenNES entwickelt.
OpenNES soll eine Informations- und Automatisierungslösung schaffen, die offen, generisch und interoperabel zur vorhandenen Infrastruktur ist und zusätzlich den aktuellen Qualitätsanforderungen hinsichtlich entspricht.


\newpage
\section{Zielsetzung}\label{sub:Zielsetzung}
Das Ziel dieser Arbeit ist es, einen \acl{PA} für die Kommunikation zwischen OpenNES fähigen
\ac{IED}s und \ac{SED}s, welche via Modbus/SunSpec kommunizieren, zu entwickeln und diesen in das Connectivity Modul zu integrieren.
Mit Hilfe des \acl{PA}s soll ein einheitliches, flexibles System zur Integration von bestehenden Standards (z.B.: IEC 61850, ModBus,...) in OpenNES geschaffen werden.
Weiterführend soll erreicht werden, dass durch die Verwendung von offenen und generischen Systemen, wie OpenNES, die Netzbetreiber eine gezielte und aktive Unterstützung bekommen um Systeme und den Stromfluss großflächig zu beeinflussen.








