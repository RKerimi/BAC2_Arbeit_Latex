\chapter{Einleitung}\label{chap:Einleitung}
\thispagestyle{standard}
\pagestyle{standard}
\renewcommand{\footrulewidth}{0.4pt}
\lfoot{\small Refik Kerimi}

Durch die Markteinführung des Smartphones hat sich unser Leben gravierend geändert. 
Nicht nur unsere Kommunikation, sondern unser Leben im Allgemeinen, ist durch dieses Gerät erleichtert worden.
Das Smartphone, Tablet und der PC sind ständig im Einsatz um Informationen abzurufen, Musik zu hören, Kontakte zu pflegen, zu telefonieren etc. Kurz nach der Erfindung des smarten Handys kam ein weiterer Markt 
hinzu, der sich parallel dazu entwickelt hat. Es wurden neue Berufe gegründet wie z.B.: der Native App Entwickler.
Native Apps werden speziell an das Betriebssystem angepasst und können somit im Gegensatz zu einer Standard Web Applikation die Ressourcen eines mobilen Gerätes optimal nutzen.\\ In der heutigen Zeit müssen Informationen jederzeit und überall für den Benutzer verfügbar sein. Unternehmen werden dadurch immer mehr gefordert, da ihre Anwendungen für jedes Gerät unabhängig von dem Betriebssystem, der Bildschirmgröße etc. funktionieren müssen. Das Ganze benötigt eigene Entwickler, die sich auf die jeweiligen Plattformen spezialisieren und auch dadurch zu höheren Entwicklungskosten führen. Um Kosten zu reduzieren und die Entwicklungen zu vereinheitlichen, startete Google ein neues Konzept, die \acl{PWA}. 
Diese Technologie soll es ermöglichen, dass sich Web-Anwendungen anfühlen wie native Anwendungen.   


\section{Motivation} \label{sub:Motivation}
Wie im vorigen Kapitel beschrieben werden native Applikationen für ein bestimmtes Betriebssystem optimiert. Diese haben den Vorteil die Hardware des Gerätes nutzen zu können und somit sind komplexere Anwendungen realisierbar. Doch diese sind für das gesamte Projekt kostspieliger, da im Gegensatz zu den Web-Anwendungen für jedes System eigene Entwickler benötigt werden.
Die \acl{PWA} (\acs{PWA}) soll die Vorteile von nativen Apps und von Webanwendungen vereinen und dem Nutzer ein Gefühl geben, dass man es mit einer auf das System angepassten Anwendung zu tun hat.  




\newpage
\section{Zielsetzung}\label{sub:Zielsetzung}
Das Ziel  ist es, mit Hilfe eines Smart Home App Prototypen die Unterschiede und Auswirkungen einer \acs{PWA} auf bestehende \acs{Web-App}s zu untersuchen.  
Dem Prototypen werden die \acs{PWA} typischen Features wie das Hinzufügen auf dem Startbildschirm, Offline arbeiten, Push-Benachrichtigungen und 
das Zugreifen auf Gerätefunktionen hinzugefügt. 
Im Laufe dieser Arbeit werden an diesen Features die Vorteile, die Nachteile, die Entwicklung, der Betrieb und die User Experience von \acl{PWA}s betrachtet.
Basis Technologien der Webentwicklung und verwendete Frameworks  \\(z.B.: \acl{JS}, ReactJS, NodeJS, Yarn,...) werden in dieser Arbeit nicht behandelt.
