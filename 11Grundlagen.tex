\chapter{Grundlagen}
\thispagestyle{standard}
\pagestyle{standard}
\renewcommand{\footrulewidth}{0.4pt}
\lfoot{\small Refik Kerimi}

Wie in Kapitel \ref{chap:Einleitung} beschrieben, hat der stetige Zuwachs von \acs{PWA}s zum Umdenken bei der Planung und beim Entwickeln von Webapplikationen geführt.
Zu beginn jedes Projektes steht die Entscheidung an, welche Technologien und Tools zur Entwicklung verwendet sollen um die bestmögliche Ergebnis zu erhalten.
Wenn die falschen Methoden gewählt werden, kann das zu gravierenden Fehlern in der Applikation führen, die sich erst mit Fortdauer der produktiven Verwendung ersichtlich werden.
Entscheidet man sich für eine Anwendung die auf das Betriebssystem zugeschnitten ist oder doch für eine plattformübergreifende Webanwendung. Beide haben Vorteile und Nachteile und diese werden im Zuge dieser Arbeit betrachtet. Der Kern der Arbeit stellt, aber die von Google entwickelten (1) Progressive Web Apps da. Die PWA() sollen den Spagat zwischen diesen beiden Anwendungen zu schaffen oder kann diese neue Form der Appentwicklung die traditionelle Technologien gar zur Gänze ablösen?



\begin{figure}[h]
	\centering
	\includegraphics[width=14cm]{BilderAllgemein/Platzhalter}\medskip
	\caption{Bildtext \cite{Quelle}}
	\label{fig:Übersicht des OpenNES Konzepts}
\end{figure}

 
\section{Geschichte Applikationsentwicklung}

\begin{itemize}
    \item 
	\item 
	\item 
\end{itemize}



\newpage
\section{Mobile APP}


\subsection{Native Apps}





\subsection{Webapplikationen}


\subsection{Progressive Web Apps}


\subsection{Unterschiede zwischen Web Apps PWA und Native Apps}

\newpage























