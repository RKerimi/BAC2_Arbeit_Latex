\chapter{Implementierung}\label{chap:Implementierung}
\thispagestyle{standard}
\pagestyle{standard}
\lfoot{\small Refik Kerimi}

\section{Umsetzung der Anforderungen}\label{sub:Umsetzung der Anforderungen}
In diesem Kapitel wird die Umsetzung der Applikation beschrieben. Die Anforderungen aus Kapitel \ref{sub:Anforderungsanalyse}
Zur Erstellung des User Interfaces wird ReactJS\footnote{https://reactjs.org/docs/getting-started.html} und als CSS Framework Semantic-UI\footnote{https://react.semantic-ui.com/introduction} verwendet, Sematic-UI soll sicherstellen das die Applikation responives verhalten aufweist und für alle Bildschirmgrößen geeignet ist. Um die Daten zu versenden, aufzurufen und zu speichern wurde das JSON Key/Value Format, die Fetch API und der Browser Cache verwendet.
Als Browser wurde der Google Chrome Version 67 verwendet.
Die nicht fertigen Funktionen wurden mit Mockups dargestellt um einen Eindruck zu vermitteln wie das ganze in Zukunft aussehen soll.

\section{Ausgewählte Programmiersprache und IDE}
Als Programmiersprache wurde \acl{JS} (\acs{JS}) ausgewählt. 
Als Entwicklungsumgebung wurde Webstorm (Version) von Jetbrains verwendet. 
Weitere verwendete Tools und Frameworks wurden im Kapitel \ref{sub:Umsetzung der Anforderungen} beschrieben.


\section{Manifest}
Das Manifest wird im Root Folder eingafügt und ist mit der Endung JSON deklariert. Der genau Pfad ist \textit{/app/manifest.json}. Wie im Kapitel \ref{sub:Manifest} schon beschrieben definiert das manifest File das Aussehen des Startbildschirm Icons, den Einstiegspunkt der App und weiter Attribute, die im Listening \ref{lst:Manifest.json} zu sehen sind:

\begin{lstlisting}[language=json, firstnumber=1, caption={Manifest in das Projekt implementieren} {\cite{Manifest}},label=lst:Manifest.json, xleftmargin=50pt]
{
  {
  "name":"PWA Smart Home",
  "short_name":"PWA_SHL_RMJ",
  "start_url":"./",
  "scope":".",
  "display":"standalone",
  "background_color":"#003399",
  "theme_color":"#3F51C5",
  "description":"Keep running with PWA",
  "dir":"ltr",
  "lang":"de-DE",
  "orientation":"portrait-primary",
  "icons":[
    {
      "src":"./static/img/light48.png",
      "type":"image/png",
      "sizes":"48x48"
    },
    {
      "src":"./static/img/light512.png",
      "type":"image/png",
      "sizes":"512x512"
    }
  ]
}
}
\end{lstlisting}


\section{Service Worker und Cache API}

%\lstset{language=Java}
%\begin{lstlisting}[caption={Status},label=lst:Status, xleftmargin=50pt]
%public enum Status {
%	Nothing,
%	Error,
%	WertGeschrieben,
%	WertEmpfangen,
%	KeineVerbindungsparameter,
%	KeinGueltigerDatentyp
%}
%\end{lstlisting}

caching Files --> welche Files sollen gachaed werden
	1. Appshell --> statisch
		2.Style, Icons, Bilder
		3. nicht zu Viel chachen
		
		
Notizen Video
	Asynchronus Code Abschitt 6 Lektion 62 4:34 
	 

\subsection{Aufbau Offline Mode}
\subsection{Implementierung}

\section{Push Notifications}
\subsection{Aufbau}
\subsection{Implementierung}

\section{Geolocation API}
\subsection{Aufbau}
\subsection{Implementierung}










