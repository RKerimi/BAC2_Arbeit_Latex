\chapter{Implementierung}\label{chap:Implementierung}
\thispagestyle{standard}
\pagestyle{standard}
\lfoot{\small Refik Kerimi}

\section{Umsetzung der Anforderungen}\label{sub:Umsetzung der Anforderungen}
In diesem Kapitel wird die Umsetzung der Applikation beschrieben. Die Anforderungen aus Kapitel \ref{sub:Anforderungsanalyse}
Zur Erstellung des User Interfaces wird ReactJS\footnote{https://reactjs.org/docs/getting-started.html} und als CSS Framework Semantic-UI\footnote{https://react.semantic-ui.com/introduction} verwendet, Sematic-UI soll sicherstellen das die Applikation responives verhalten aufweist und für alle Bildschirmgrößen geeignet ist. Um die Daten zu versenden, aufzurufen und zu speichern wurde das JSON Key/Value Format, die Fetch API und der Browser Cache verwendet.
Als Browser wurde der Google Chrome Version 67 verwendet.
Die nicht fertigen Funktionen wurden mit Mockups dargestellt um einen Eindruck zu vermitteln wie das ganze in Zukunft aussehen wird.

\section{Ausgewählte Programmiersprache und IDE}
Als Programmiersprache wurde \acl{JS} (\acs{JS}) ausgewählt. 
Als Entwicklungsumgebung wurde Webstorm (Version) von Jetbrains verwendet. 
Weitere verwendete Tools und Frameworks wurden im Kapitel \ref{sub:Umsetzung der Anforderungen} beschrieben.


\section{Manifest}
\subsection{Aufbau}
\subsection{Implementierung}



\section{Service Worker und Cache API}

%\lstset{language=Java}
%\begin{lstlisting}[caption={Status},label=lst:Status, xleftmargin=50pt]
%public enum Status {
%	Nothing,
%	Error,
%	WertGeschrieben,
%	WertEmpfangen,
%	KeineVerbindungsparameter,
%	KeinGueltigerDatentyp
%}
%\end{lstlisting}

caching Files --> we´lche Files sollen gachaed werden
	1. Appshell --> statisch
		2.Style, Icons, Bilder
		3. nicht zu Viel chachen
		
		
Notizen Video
	Asynchronus Code Abschitt 6 Lektion 62 4:34 
	 

\subsection{Aufbau Offline Mode}
\subsection{Implementierung}

\section{Push Notifications}
\subsection{Aufbau}
\subsection{Implementierung}

\section{Geolocation API}
\subsection{Aufbau}
\subsection{Implementierung}










