\chapter*{Allgemeine Informationen}
\thispagestyle{plain}
\pagestyle{plain}
\renewcommand{\footrulewidth}{0.4pt}
\lfoot{\small Refik Kerimi}

\begin{tabular}{p{0.3\textwidth}p{0.65\textwidth}}

Vor- und Zuname: & \Author \\*[0.2cm]
Institution: & Fachhochschule Salzburg GmbH \\*[0.2cm]
Studiengang: & Informationstechnik \& System-Management \\*[0.2cm]
Titel der Bachelorarbeit: & \Title \\*[0.2cm]
Schlagwörter: & \Keywords  \\*[0.2cm]
Betreuer an der FH: & \Advisor

\end{tabular}

\newpage

\section*{\Large\bfseries Kurzfassung}
Mit der Einführung der Betriebsysteme IOS im Jahr 2007 und Android ein Jahr später, wurden die ersten Smartphones zu unseren täglichen Begleitern im Alltag.
Im Gegensatz zu früheren Smartphoneherstellern setzten Apple und Google auf die benutzerfreundliche Bedienung der Geräte.
Die Einführung der App Stores brachte einen großen Schub in der Entwicklung der Nativen Apps. 
Bei den \acl{Web-App}s wurde die Entwicklung auch vorangetrieben und immer bessere Browser kamen auf den Markt.
In dieser Bachelorarbeit wird die neue Technologie von Google behandelt, die \acl{PWA}. Diese gibt den \acl{Web-App}s eine höhere Benutzerfreundlichkeit.
Anhand der Implementierung von \acs{PWA}-Features wie dem Service Worker, der Manifest-Datei, der Push Notifikation und dem Zugriff auf die Geolocation API, in eine Standard \acs{Web-App} sollen die Vorteile, Nachteile und Usability von \acs{PWA}s genauer geprüft und getestet werden. Durch diese Features sollen die bisher vermissten Features einer nativen App, z.B.: schneller Zugriff durch ein Icon auf dem Startbildschirm, Offlinefunktionen oder Benachrichtigung der User in die Browserapplikation integriert werden.
Außerdem wird evaluiert ob die \acs{PWA} die nativen Apps zur Gänze ablösen kann. 



\section*{\Large\bfseries Abstract}
With the introduction of operating systems IOS in 2007 and Android a year later, the first smartphones became our daily companions in everyday life.
Unlike previous smartphone the manufacturers, Apple and Google relied on the user-friendly operation of the devices.
The introduction of the App-Shops brought a big boost in the development of native apps.
The development of the \acl{Web-App} has also progsressed and ever better browsers have hit the market.
This Bachelor Thesis covers Google's new technology, the \acl{PWA} which gives   the \acl{Web-App} a better usability.
Implementing \acs{PWA}-features such as the service worker, manifest file, push notification, and accessing the geolocation API in a standard \acs{Web-App} are intended to demonstrate the advantages, disadvantages, and usability, be checked and tested by \acs{PWA}s. These features are purpose to integrate the previously missing features of a native app, such as quick access by an icon on the home screen, offline functions or notification of users in the browser application.
In addition, it is evaluated whether the \acs{PWA} can completely replace the native apps.

