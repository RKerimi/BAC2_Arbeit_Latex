\chapter{Zusammenfassung und Ausblick}
\thispagestyle{standard}
\pagestyle{standard}
\renewcommand{\footrulewidth}{0.4pt}
\lfoot{\small Refik Kerimi}

%Der Offline Modus funktioniert leider nicht immer, da des öfteren problem bei den %Abhängigkeiten im Projekt gibt. Aber diese Fuktion zahlt sich alleine schon aus, 
%da man mit einer schlechten Datenrate schon sehr schnell die App aufrufen kann.
In diesem Projekt hat sich gezeigt, dass die \acs{PWA} die Nativen Applikationen nicht zur Gänze ablösen kann. Es ist sicherlich so, dass die \acs{PWA} eine Verbesserung der \acs{Web-App}s ist, aber sie  arbeitet trotzdem nicht so gut wie eine für das Betriebssystem konzipierte Applikation.
Probleme gab es beim offline arbeiten durch die Verwendung des \acs{JS}-Frameworks RactJS, durch die modulare Bauweise wurden die gecachten Files nicht immer richtig geladen und es ist mehr oder weniger Zufall ob die Offlinefunktion tatsächlich wie gewünscht läuft. Doch auch wenn diese Funktion verbesserungswürdig ist, stellt sie trotzdem eine Verbesserung der Applikation dar, da bei geringer Internetgeschwindigkeit die App trotzdem sehr schnell und flüssig funktioniert. Bei der Entwicklung mit \acs{HTML}5, \acs{CSS} und \acs{JS} sollte es besser funktionieren. Dies wurde in dieser Arbeit, aber nicht behandelt. 
Der Service Worker ist sehr einfach zu implementieren und arbeitet ohne irgendwelche Behinderungen im Hintergrund.
Die Push Notification ist eine große Bereicherung für die Browserapplikation, da dadurch Apps wie z.B.: Smart Home Apps, Sozial Media Apps, uvm. die User viel besser erreichen können. Durch das Hinzufügen des Icons auf dem Startbildschirm wird eine zusätzlich eine höhere Kundenbindung erreicht. Der Vorteil dabei ist, dass der Kunde nicht mehr über den Browser, durch die Eingabe der URL auf die Webseite gelangt sondern über seinen Startbildschirm wie bei einer Nativen Applikation. Ein Nachteil ist sicherlich die Browserkompatilität. Die \acs{PWA} wird nicht von allen Herstellern unterstützt und einige Features sind nur auf Android Geräten verfügbar. 
\\Im großen und ganzen denke ich ist die \acs{PWA} eine sehr aufregende Form der \acs{Web-App}s und man wird in Zukunft sicher noch einiges davon hören.
Als ich diese Arbeit Anfang des Jahres begann war die Community noch sehr klein, doch innerhalb von ein paar Monaten hat sie sich verändert und ist sehr gewachsen. Es finden sich viele nützliche Tipps online, die es ermöglichen in sehr kurzer Zeit aus einer bestehenden App eine \acs{PWA} zu erstellen. 