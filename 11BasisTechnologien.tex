\chapter{Basistechnologien}
\thispagestyle{standard}
\pagestyle{standard}
\renewcommand{\footrulewidth}{0.4pt}
\lfoot{\small Markus Haas}

\section{Manifest}

In diesem Kapitel werden die Besonderheiten einer PWA erklärt.

Der Protokolladapter ist für Modbus TCP entwickelt worden, deshalb wird auch vorwiegend diese Version erklärt. Entwickelt wurde das Protokoll aber ursprünglich für die serielle Schnittstelle. Daher wird zuerst das Protokoll für die serielle und anschließend die Erweiterung für TCP beschreiben.

\subsection{}



\subsection{Grundlagen}


\begin{itemize}
    \item Modbus TCP/IP
\end{itemize}


\subsection{Protokollaufbau}
\label{sec:Protokollaufbau}


\subsection{Modbus Datenmodell}

\label{tab:modbusDataModel}

\subsection{Funktionscodes} 
\label{sec:Funktionscodes}

\subsubsection{Lese Holding Registers}
\label{sec:LeseHoldingRegisters}

\subsubsection{Schreibe Multiple Registers}
\label{sec:SchreibeMultipleRegister}

\subsection{ModBus TCP/IP}



\section{SunSpec}

\subsection{Übersicht}


\subsection{SunSpec Informationsmodell}
\label{sec:SunsSpecInformationsmodell}


\subsection{SunSpec Datentypen}


\subsection{Register Mapping}

